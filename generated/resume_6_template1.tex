%-------------------------
% Resume in Latex — Jake Template (Dynamic Version)
%------------------------

\documentclass[letterpaper,11pt]{article}

\usepackage{latexsym}
\usepackage[empty]{fullpage}
\usepackage{titlesec}
\usepackage{marvosym}
\usepackage[usenames,dvipsnames]{color}
\usepackage{verbatim}
\usepackage{enumitem}
\usepackage[hidelinks]{hyperref}
\usepackage{fancyhdr}
\usepackage[english]{babel}
\usepackage{tabularx}
\usepackage{fontawesome5}
\usepackage{multicol}
\input{glyphtounicode}

\setlength{\multicolsep}{-3pt}
\setlength{\columnsep}{-1pt}

%----------Margins----------
\pagestyle{fancy}
\fancyhf{}
\fancyfoot{}
\renewcommand{\headrulewidth}{0pt}
\renewcommand{\footrulewidth}{0pt}

\addtolength{\oddsidemargin}{-0.6in}
\addtolength{\evensidemargin}{-0.5in}
\addtolength{\textwidth}{1.19in}
\addtolength{\topmargin}{-.7in}
\addtolength{\textheight}{1.4in}

\urlstyle{same}
\raggedbottom
\raggedright
\setlength{\tabcolsep}{0in}

%----------Section Formatting----------
\titleformat{\section}{
  \vspace{-4pt}\scshape\raggedright\large\bfseries
}{}{0em}{}[\color{black}\titlerule \vspace{-5pt}]

\pdfgentounicode=1

%----------Custom Commands----------
\newcommand{\resumeItem}[1]{\item\small{{#1 \vspace{-2pt}}}}
\newcommand{\resumeSubheading}[4]{
  \vspace{-2pt}\item
  \begin{tabular*}{1.0\textwidth}[t]{l@{\extracolsep{\fill}}r}
    \textbf{#1} & \textbf{\small #2} \\
    \textit{\small #3} & \textit{\small #4} \\
  \end{tabular*}\vspace{-7pt}
}
\newcommand{\resumeProjectHeading}[2]{
  \item
  \begin{tabular*}{1.0\textwidth}{l@{\extracolsep{\fill}}r}
    \small#1 & \textbf{\small #2} \\
  \end{tabular*}\vspace{-7pt}
}

\newcommand{\resumeSubHeadingListStart}{\begin{itemize}[leftmargin=0in,label={}]} 
\newcommand{\resumeSubHeadingListEnd}{\end{itemize}}
\newcommand{\resumeItemListStart}{\begin{itemize}}
\newcommand{\resumeItemListEnd}{\end{itemize}\vspace{-5pt}}

%=====================================
%         RESUME STARTS
%=====================================

\begin{document}

%----------HEADER----------
\begin{center}
    {\Huge \scshape First Last} \\ \vspace{1pt}
    123 Street Name, Town, State 12345 \\ \vspace{1pt}
    \small 
    \raisebox{-0.1\height}\faPhone\  123-456-7890 
    ~ \href{mailto:email@gmail.com}{\raisebox{-0.2\height}\faEnvelope\ \underline{email@gmail.com}}
    ~ \href{linkedin.com/in/username}{\raisebox{-0.2\height}\faLinkedin\ \underline{Linkedin}}
    ~ \href{github.com/username}{\raisebox{-0.2\height}\faGithub\ \underline{Github}}
    \vspace{-8pt}
\end{center}

%-----------EDUCATION-----------
\section{Education}
\resumeSubHeadingListStart

\resumeSubHeadingListEnd

%-----------COURSEWORK-----------
\section{Relevant Coursework}
\begin{multicols}{4}
\begin{itemize}[itemsep=-5pt,parsep=3pt]
\item Data Structures 
\item Oops
\item DBMS
\item Software Engg
\item Operating Systems
\end{itemize}
\end{multicols}
\vspace*{2.0\multicolsep}

%-----------EXPERIENCE-----------
\section{Experience}
\resumeSubHeadingListStart

\resumeSubheading
{Electronics Company}{May 2025 -- July 2026}
{Software Engineer }{City, State}
\resumeItem{ Developed a service to automatically perform a set of unit tests daily on a product in development in order to decrease
time needed for team members to identify and fix bugs/issues.
• Incorporated scripts using Python and PowerShell to aggregate XML test results into an organized format and to load
the latest build code onto the hardware, so that daily testing can be performed.
• Utilized Jenkins to provide a continuous integration service in order to automate the entire process of loading the latest
build code and test files, running the tests, and generating a report of the results once per day.
• Explored ways to visualize and send a daily report of test results to team members using HTML, Javascript, and CSS.}


\resumeSubheading
{Startup, Inc }{May 2019  -- August 2019}
{Front End Developer Intern }{City, State}
\resumeItem{Assisted in development of the front end of a mobile application for iOS/Android using Dart and the Flutter framework.
• Worked with Google Firebase to manage user inputted data across multiple platforms including web and mobile apps.
• Collaborated with team members using version control systems such as Git to organize modifications and assign tasks.
• Utilized Android Studio as a development environment in order to visualize the application in both iOS and Android.}

\resumeSubHeadingListEnd

%-----------PROJECTS-----------
\section{Projects}
\resumeSubHeadingListStart

\resumeProjectHeading
{Gym Reservation Bot (\textit{May 2025})}{}
\resumeItem{Tech: Python, Selenium, Google Cloud Console }
\resumeItem{Developed an automatic bot using Python and Google Cloud Console to register myself for a timeslot at my school gym.
Implemented Selenium to create an instance of Chrome in order to interact with the correct elements of the web page.
Created a Linux virtual machine to run on Google Cloud so that the program is able to run everyday from the cloud.
Used Cron to schedule the program to execute automatically at 11 AM every morning so a reservation is made for me}


\resumeProjectHeading
{Ticket Price Calculator App (\textit{Nov 2020})}{}
\resumeItem{Tech: Java, Android Studio November 2020}
\resumeItem{Created an Android application using Java and Android Studio to calculate ticket prices for trips to museums in NYC.
Processed user inputted information in the back-end of the app to return a subtotal price based on the tickets selected.
Utilized the layout editor to create a UI for the application in order to allow different scenes to interact with each other.}


\resumeProjectHeading
{Transaction Management GUI  (\textit{May 2027})}{}
\resumeItem{Tech: ava, Eclipse, JavaFX }
\resumeItem{Designed a sample banking transaction system using Java to simulate the common functions of using a bank account.
Used JavaFX to create a GUI that supports actions such as creating an account, deposit, withdraw, list all acounts, etc.
Implemented object-oriented programming practices such as inheritance to create different account types and databas}

\resumeSubHeadingListEnd

%-----------SKILLS-----------
\section{Technical Skills}
\begin{itemize}[leftmargin=0.15in,label={}]
  \small{
  \item{
    \textbf{Languages}{: Python, Java, C, HTML/CSS, JavaScript, SQL} \\
    \textbf{Developer Tools}{: VS Code, Eclipse, Google Cloud Platform, Android Studio} \\
    \textbf{Technologies/Frameworks}{: Linux, Jenkins, GitHub, JUnit, WordPress} \\
  }}
\end{itemize}
\vspace{-16pt}

%-----------EXTRA-----------
\section{Leadership / Extracurricular}
\resumeSubHeadingListStart

\resumeSubheading
{Fraternity }{May 2025 -- July 2025}
{President }{}
\resumeItem{Achieved a 4 star fraternity ranking by the Office of Fraternity and Sorority Affairs (highest possible ranking).
Managed executive board of 5 members and ran weekly meetings to oversee progress in essential parts of the chapter.
Led chapter of 30+ members to work towards goals that improve and promote community service, academics, and unity.
}

\resumeSubHeadingListEnd

\end{document}
